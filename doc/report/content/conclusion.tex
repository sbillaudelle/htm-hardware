Implementing machine intelligence algorithms as spiking neural networks and porting them to a neuromorphic hardware platform presents high demands in terms of precision and scalability.

We have shown in this paper, that \glspl{htm} can be successfully modeled in dynamic simulations. The basic functionality of spatial pooler and temporal memory networks could be reproduced based on \gls{adex} neurons. In theory, the proof of concept networks can be easily transferred to the \gls{hmf}, since the high-level software interfaces are designed to be interchangable. Of course, emulating the models on the actual hardware platform will bring up a new set of challenges.

Adapting the \gls{htm}'s plasticity rules to the features available on the \gls{hmf} has turned out to be highly nontrivial. The learning rules could not be replicated with the current implementation of classic \gls{stdp}. As a freely programmable microprocessor directly embedded into the neuromorphic core, the \gls{ppu} will in future extend the flexibility of the system's plasticity mechanisms. While first prototype chips are already available, the processor's actual capabilities have to be outlined in a production environment. Further investigation is required to map out the plans for a possible implementation of the \gls{hmf} update rules on the \gls{ppu}.

Analog neuromorphic hardware is susceptible to transistor mismatches due to e.g. dopand fluctuations in the production process. A careful calibration of the individual neurons is required to compensate for these variations. Due to the complexity of the problem and the high number of interdependent variables, a perfect calibration is hard to accomplish. Therefore, network models are required to be tolerant regarding certain spatial, and trial-to-trial variations on the computing substrate. Carrying out additional Monte Carlo simulations with slightly randomized parameters is important to investigate the robustness of the presented networks.

Finally, a multicompartmental neuron model is planned for a later version of the neurmorphic platform. Making use of this extended feature set will significantly increase the level of biophysical detail. This will account for the detailed dendritic model used in \glspl{htm} and help to stay closer to the whitepaper as well as the reference implementation.

Besides paving the road towards a highly accelerated execution of \gls{htm} models, the \gls{hmf} offers a high level of detail in its neuron implementation. With the multicompartmental extension and a flexible plasticity framework, the platform might prove valuable as a tool for further low-level research on \gls{htm} theories.
