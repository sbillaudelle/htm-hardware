Mammals -- and especially human beings -- are able to process diverse sensory inputs, learn and recognize complex spatial and temporal patterns, and generate behaviour based on current context and previous experiences. While computers are efficient in numerical calculations, they fall short in cognitive tasks. Studying the brain and the neocortex in particular is important to develop new algorithms closing the gap between living intelligent organisms and artificial systems. Numenta is a company dedicated to developing such algorithms and at the same time investigating the neocortex.

Efficiently simulating large-scale neural models in software still is a challenge. Different techniques for speeding up the execution of such implementations exist. At the same time, dedicated hardware platforms are being developed. Digital neuromorphic hardware often feature highly parallelized processing architectures and optimized connectivity. Analog systems however, emulate the neuron's behavior in electronic microcircuits. In the following, first efforts in porting \gls{htm} networks to the \gls{hmf} mixed-signal neuromorphic platform will be presented.


\subsection{Hirarchical Temporal Memory}

\gls{htm} represents a set of concepts and algorithms for machine intelligence based on neocortical principles \citep{numenta2011htm}. It is designed to learn \emph{spatial as well as temporal patterns} and generate predictions from previously seen sequences. It features \emph{continuous learning} and operates on streaming data. An \gls{htm} network consists of one or multiple hirarchically arranged \emph{regions}. The latter consist of neurons organized in columns. The functional principle is captured in two algorithms which are layed out in detail in the original whitepaper \citep{numenta2011htm}. The following paragraphs are intended as an introductory overview only.

%\subsubsection{Spatial Pooler}
\label{sec:spatial_pooler_properties}

The \emph{spatial pooler} is designed to map a binary input vector to a set of columns. By recognizing previously seen input data, it increases stability and reduces the systems susceptibility for noise. Its behaviour can be characterized by a collection of properties:

\begin{enumerate}
	\item\label{enm:spatial_pooler_sparsity} The colums's activity is spatially sparse. Typically, 40 out of 1,000 colums are active, which is equivalent to a sparsity of \SI{4}{\%}. The number of active columns is constant for each time step and does not depend on the input sparsity.
	\item\label{enm:spatial_pooler_minimum} A column must receive a minimum input of e.g. 15 to become active.
	\item\label{enm:spatial_pooler_selection} The spatial pooler selects the $k$ columns which receive the most input. In case of a tie between two columns, the active column is selected randomly.
	\item\label{enm:spatial_pooler_overlap} Similar input results in similar activity patterns. For disjunct stimuli, the network yields activity patterns with low overlap counts.
\end{enumerate}

%\subsubsection{Temporal Memory}

The \emph{temporal memory} operates on single cells and further processes the spatial pooler's output. Individual cells receive stimulus from other neurons on their distal dendrites. This additional input provides a temporal context. By modifying a cell's distal connectivity, temporal sequences can be learned and predicted.

\subsection{Heidelberg Neuromorphic Computing Platform}

The \gls{hmf} is a mixed-signal neuromorphic platform developed at the Kirchhoff-Institute for Physics in Heidelberg and the TU Dresden and funded by the \gls{bss} and the \gls{hbp}. Core of the system is the \gls{hicann} chip.

\gls{hicann} features 512 analog point neurons -- or \emph{dendritic membranes} -- which can be stimulated via two independent synaptic inputs. As a default, the latter are configured for excitatory and inhibitory stimuli, respectively. However, they can be set up to represent e.g. two excitatory inputs with different synaptic time constants or reversal potentials. Multiple dendritic membrane circuits can be connected to form a larger neuron and allow for a higher number of synaptic inputs. Each dendritic membrane circuit can be stimulated via 113 synapses per synaptic input.

The system's timescale results from the intrinsic time constants of the hardware neurons. The \gls{hmf} operates with a speed up of $10^4$ compared to biological real-time.

\subsection{Spiking Neuron Model}

There exist several, fundamentally different methods of varying complexity for modelling neural networks. While the reference implementation of the \gls{htm} networks is based on first generation, \emph{digital} neurons \citep{nupic}, most analog \gls{vlsi} systems implement third generation, \emph{spiking} neurons.

The \gls{hmf} features \gls{adex} neurons \citep{brette2005adaptive}. This model was found to correctly predict approximately \SI{96}{\%} of the spike times of a Hodgkin-Huxley-type model neuron and about \SI{90}{\%} of the spikes recorded from a cortical neuron \citep{jolivet2008quantitative}.

For the following simulations, \gls{adex} neurons with conductance-based synapses were used. This allows for the implementation of e.g. shunting inhibition.
